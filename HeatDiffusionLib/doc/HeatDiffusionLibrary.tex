\documentclass[a4paper,10pt]{article}
\usepackage[utf8]{inputenc}

%opening
\title{Heat Diffusion Library}
\author{Version 1.0}
\date{}

\begin{document}

\maketitle

\section{Introduction}
The Heat Diffusion Library is a C++ library for the computation of Heat Diffusion descriptors and distances. The library is intended to help 
in the development of Heat Diffusion algorithms in C++ applications. It contains implementations for:

\begin{itemize}
 \item Discrete Laplacian matrices
 \item Eigendecomposition of Laplacian
 \item Descriptors: Heat Kernel Signatures and Wave Kernel Signatures
 \item Distances: Diffusion distance, Conmute-time distance and Biharmonic distance
\end{itemize}

\section{Compilation}
The library can be easily compiled using CMake. The package contains a CMake configuration file for building. The commands to compile 
the library as a shadow build are

\begin{verbatim}
    >> mkdir build; cd build
    >> cmake ..
    >> make
\end{verbatim}

These commands will create a shared library.

\subsection{Dependencies}
The Heat Diffusion Library has a few dependencies as follows:

\begin{itemize}
 \item Eigen. The provided CMake configuration file will search for an installation of Eigen in your system.
 \item Nanoflann. Provided in the library as Third-party
 \item Arpack++. Provided in the library as Third-party
\end{itemize}

\section{Usage}
To show the usage of the library in a C++ application, we provide a test program. In the test/ folder, you can find a program that reads a
OFF file (3D mesh) and computes the Heat Kernel Signatures and Wave Kernel Signatures. The program finally saves the descriptors in files. 

The test program also contains a CMake configuration file to show how to use the Heat Diffusion Library in an application. To properly compile
the test program, you need to have SuperLU, Blas, Lapack and Arpack installed in your system. We provide a static SuperLu library as Third-party.

To execute the test program, we provide a OFF file in the test/data/ folder.

\section{Documentation}
The documentation of the classes are provided in the doc/ folder. It contains the documentation in html and latex.


\end{document}
